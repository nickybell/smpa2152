\documentclass[12pt,letterpaper]{article}
\usepackage[margin=1in]{geometry}
\usepackage{multirow}
\usepackage[dvipsnames]{xcolor}
\usepackage[colorlinks,allcolors=Blue]{hyperref}
\usepackage[rm,light]{roboto}
\usepackage[T1]{fontenc}
\usepackage{soul}
\usepackage{array}
\newcolumntype{C}[1]{>{\centering\arraybackslash}p{#1}}
\usepackage{ltablex}
\usepackage{arydshln}

\frenchspacing
\setlength{\parindent}{0pt}
\setlength{\parskip}{1em}

\begin{document}

\begin{center}
  \large
  \textbf{Syllabus\\
    \bigskip
    SMPA 2152: Data Analysis for Journalism and Political Communication
  (Spring 2026)}
\end{center}

\begin{tabularx}{\textwidth}{l>{\raggedright\arraybackslash}X}
  Meeting Times: & Mondays and Wednesdays, 8:00-9:15am \\
  \\
  Classroom: & MPA 307 \\
  \\
  Professor: & Nicholas Bell, Ph.D. (he/him) \\
  & \href{mailto:nicholasbell@gwu.edu}{nicholasbell@gwu.edu} \\
  \\
  Office Hours: & Tuesdays 5:45 - 6:45pm \\
  & MPA 425 \\
  & \\
  & By appointment only. Appointments must be made at least three
  hours prior to office hours. The scheduling link is
  \href{https://tinyurl.com/smpa2152officehours}{https://tinyurl.com/smpa2152officehours}
  \\
  & \\
  & If you would like to meet outside of office hours, please email
  me to schedule an appointment. I try to respond to emails by the
  end of the next business day (M-F).\\
  \\
  \hline
\end{tabularx}

\subsection*{Course Description}

Data has been democratized. More data is available to the ordinary
person than ever before, and leaders in every industry -- including
journalism and political communication -- want to join the big data
revolution. However, most of us lack the data literacy skills to make
good use of these resources, and this can lead to the misapplication
and misuse of data. To fully leverage the promise of big data, we
must become familiar with the basic challenges inherent in data
analysis and how to overcome them. This course is an introduction to
the principles and practices of data analysis. The goal is for
students to become responsible consumers and producers of data.
Students will learn how to critically evaluate claims derived from
data. Students will also learn how to ethically present data in
compelling and persuasive ways to non-expert audiences. This class
includes a special discussion of political polling, which is widely
used in journalism and political communication but has come under
increasing scrutiny in recent years. Students require only a basic
aptitude in numeracy (e.g. percentages and averages) for this course. \par

In addition to developing data literacy, students will be introduced
to the \textit{R} programming language. There are many advantages to
learning \textit{R}: it is free and open-source, meaning that
developers are continually releasing new tools to make coding easier;
it is widely used by news organizations and researchers around the
world; and \textit{R} is one of the most powerful programming
languages for statistical analysis. Students will learn data literacy
by applying the same tools and techniques used by professional data scientists.

\subsection*{Learning Objectives}

\begin{enumerate}
  \item You will be able to assess the pragmatic and ethical issues
    in collecting, manipulating, and analyzing data, known as ``data literacy.''
  \item You will be able to obtain publicly-available data and
    perform basic manipulations on that data using the programming
    language \textit{R}.
  \item You will be able to visualize and present data in accurate
    and persuasive ways.
  \item You will be familiar with the statistical concepts of
    sampling, uncertainty, hypothesis testing, and linear regression
    and how to conduct basic statistical analyses in \textit{R}.
\end{enumerate}
\vspace{-.5em}

\subsection*{Course Materials}

We will be using \href{https://colab.research.google.com/}{Google
Colab} in this course. Currently, Google is offering a free
\href{https://colab.research.google.com/signup}{Colab Pro}
subscription for one year to university students. Although a
subscription is not strictly required, it will ensure that Colab is
always available to you (free users may experience usage limits in
order to keep the service available to subscribed users).\\
\\
In addition, I recommend purchasing a
\href{https://chatgpt.com/plans/plus/}{ChatGPT Pro},
\href{https://claude.com/pricing}{Claude Pro}, or
\href{https://gemini.google/students/}{Gemini Pro} plan. Google is
currently offer one year free of Gemini Pro for university students.

\subsection*{Time Required}

This course requires 2.5 hours of direct instruction and a minimum of
5 hours of independent learning per week for a combined minimum total
of 7.5 hours per week or 112.5 hours per semester.

\subsection*{How the Course Will Work}

This course is primarily an introduction to the principles and
practices of data analysis. Most class days will consist of a lecture
on a topic related to consuming and producing data. However, there
are seven class days reserved as ``Lab Days'' for students to be
introduced to the R statistical programming language. The purpose of
learning R is so that students can apply the principles learned
during lecture to real-world data and gain exposure to the types of
tasks undertaken by professional data analysts in their chosen career field.

To prepare for the Lab Days, students should watch the associated
video lecture \textbf{before} attending that class. Failure to watch
the video lecture will significantly hinder the student's ability to
complete the lab assignment. During Lab Days, the instructor will
provide a brief review of the material covered during the video
lecture, and then be available to answer student questions as they
work through their lab assignment in class. Students will then
complete the lab assignment on their own for credit.

In addition, students will participate in a class project to conduct
a survey of their fellow GW students. Each student will be assigned a
role (based on their preference) in designing, executing, and
analyzing the survey, and their participation will make up a portion
of their final course grade. In addition, there is a Lab Day
assignment which uses the survey data collected by students.

On January 26, students will be provided instructions to complete
CITI Research Ethics training. This training is required by the
university in order to be able to participate in the class survey
project. Failing to complete CITI training will result in a 0 grade
for the class project and the associated lab assignment.

There is a final exam in this course (see below).

\subsection*{Assessment}

Your course grade is calculated as your grade on each of the
following course components weighted by:

\begin{tabular}{l|l}
  Lab assignments & 35\% \\
  \hline
  Class project & 20\% \\
  \hline
  Final exam & 35\% \\
  \hline
  Attendance & 10\%
\end{tabular}
\newpage
Course grades are converted into letter grades according to the
following rubric:\\
\\
93-100 = A (4.0 GPA points)\\
90-92 = A- (3.7 GPA points)\\
87-89 = B+ (3.3 GPA points)\\
83-86 = B (3.0 GPA points)\\
80-82 = B- (2.7 GPA points)\\
77-79 = C+ (2.3 GPA points)\\
73-76 = C (2.0 GPA points)\\
70-72 = C- (1.7 GPA points)\\
67-69 = D+ (1.3 GPA points)\\
63-66 = D (1.0 GPA points) \\
60-62 = D- (0.7 GPA points)

\subsection*{Lab Assignments}

There are seven lab assignments in this course that will ask you to
apply the R skills that you learn in the video lectures. I will be
available to help you get started on your lab assignment during class
Lab Days, and then you will finish the lab assignment on your own.
You may complete these assignments on your own or in collaboration
with other students. This means that you may work together to write
code and/or solve problems. Do not split up the questions or combine
independent work. \textbf{It is a violation of the academic integrity
  policy to submit any code to which you did not contribute as your
own.} If you work with other students, please indicate their names at
the top of your submission. Each student must submit an assignment on
Blackboard. \par

Assignments are typically due by 11:59pm Eastern of the day following
the lab day. Late assignments are deduced 25\% per day. \par

\subsection*{Final Exam}

The final exam will be held at the University-prescribed time on
Blackboard and may be completed from the location of your choosing.
The final exam will consist of true-false/multiple choice questions,
short answer questions, and an essay prompt. There is no coding
required for the final exam.

\subsection*{Attendance}

Attendance is mandatory. Each student is permitted two excused
absences; the third and subsequent absences are only permitted due to
due to illness, family emergencies, University-scheduled events, and
other unusual circumstances. If you will be absent for the third or
subsequent time, you must email me and let me know the reason for
your absence. You do not need to provide proof of your reason for
missing class, but misrepresentation of your reason for excusal is a
violation of the
\href{https://students.gwu.edu/code-student-conduct}{Code of Student
Conduct}. \par

Your attendance grade is the percentage of class meetings with an
unexcused absence deducted from 100 (rounded up). For example, if you
have two unexcused absences, your attendance grade is
$100-((2/27)*100) = 93$. \par

\textbf{University Policy on Observance of Religious Holidays} \par

Students must notify faculty during the first week of the semester in
which they are enrolled in the course, or as early as possible, but
no later than three weeks prior to the absence, of their intention to
be absent from class on their day(s) of religious observance. If the
holiday falls within the first three weeks of class, the student must
inform faculty in the first week of the semester. For details and
policy, see ``Religious Observance Policy'' at
\href{https://provost.gwu.edu/policies-procedures-and-guidelines}{provost.gwu.edu/policies-procedures-and-guidelines}.

\subsection*{Support for Students with Disabilities}

Any student who may need an accommodation based on the impact of a
disability should contact the Office of Disability Support Services
(DSS) to inquire about the documentation necessary to establish
eligibility and to coordinate a plan of reasonable and appropriate
accommodations. DSS is located in Rome Hall, Suite 102. For
additional information, please call DSS at 202-994-8250, or consult
\href{https://disabilitysupport.gwu.edu/}{disabilitysupport.gwu.edu/}.

\subsection*{Academic Integrity}

Academic integrity is an essential part of the educational process,
and all members of the GW community take these matters very
seriously. As the instructor for this course, my role is to provide
clear expectations and uphold them in all assessments. Violations of
academic integrity occur when students fail to cite research sources
properly, engage in unauthorized collaboration, falsify data, and
otherwise violate the
\href{https://students.gwu.edu/code-academic-integrity}{Code of
Academic Integrity}. If you have any questions about whether or not
particular academic practices or resources are permitted, you should
ask me for clarification. If you are reported for an academic
integrity violation, you should contact Conflict Education and
Student Accountability (CESA), formerly known as Student Rights and
Responsibilities (SRR), to learn more about your rights and options
in the process. Consequences can range from failure of assignment to
expulsion from the university and may include a transcript notation.
For more information, please refer to the
\href{https://students.gwu.edu/cesa}{CESA website}, email
\href{mailto:cesa@gwu.edu}{cesa@gwu.edu}, or call 202-994-6757.

\subsection*{Course Policy on Generative AI}

Generative Artificial Intelligence (GAI) tools like Gemini, ChatGPT,
and Claude are increasingly used in academic and professional
settings to make certain non-analytical tasks more efficient.
Therefore, this course permits the use of GAI tools on \textbf{code}
submitted for evaluation. However, the use of GAI tools for
\textbf{written text} (e.g., exposition, analysis, etc.) is not permitted.

By submitting written work for evaluation in this course, you
represent it as your own intellectual product. Impermissible use of
GAI tools for written work submitted for evaluation constitutes
cheating under the GW's
\href{https://students.gwu.edu/code-academic-integrity}{Code of
Academic Integrity}. If you have any questions about the application
of this policy (or any other questions about academic integrity in
this course), please email the instructor.

\subsection*{Class Recordings and Use of Electronic Course Materials}

Class meetings will be audio/video recorded and made available to
other students in this course. As part of your participation in this
course, you may be recorded. If you do not wish to be recorded,
please contact me during the first week of class to discuss
alternative arrangements. \par

Students are encouraged to use electronic course materials, including
recorded class sessions, for private personal use in connection with
their academic program of study. Electronic course materials and
recorded class sessions should not be shared or used for non-course
related purposes unless express permission has been granted by the
instructor. Students who impermissibly share any electronic course
materials are subject to discipline under the
\href{https://studentconduct.gwu.edu/code-student-conduct}{Code of
Student Conduct}. Please contact the instructor if you have questions
regarding what constitutes permissible or impermissible use of
electronic course materials and/or recorded class sessions. Please
contact \href{https://disabilitysupport.gwu.edu/}{Disability Support
Services} if you have questions or need assistance in accessing
electronic course materials.

\subsection*{Additional Resources for Students}

\textbf{Academic Support}

\begin{itemize}
  \item \textbf{Academic Commons} \\
    Academic Commons is the central location for academic support
    resources for GW students. To schedule a peer tutoring session
    for a variety of courses visit
    \href{https://go.gwu.edu/tutoring}{go.gwu.edu/tutoring}. Visit
    \href{https://academiccommons.gwu.edu}{academiccommons.gwu.edu}
    for study skills tips, finding help with research, and connecting
    with other campus resources. For questions email
    \href{mailto:academiccommons@gwu.edu}{academiccommons@gwu.edu}.
  \item \textbf{Writing Center} \\
    GW’s Writing Center cultivates confident writers in the
    University community by facilitating collaborative, critical, and
    inclusive conversations at all stages of the writing process.
    Working alongside peer mentors, writers develop strategies to
    write independently in academic and public settings. Appointments
    can be booked online at
    \href{https://gwu.mywconline.com}{gwu.mywconline.com}.
\end{itemize}

\textbf{Health and Wellness Support}

\begin{itemize}
  \item \textbf{Disability Support Services} \\
    202-994-8250 \\
    Any student who may need an accommodation based on the potential
    impact of a disability should contact Disability Support Services
    at
    \href{https://disabilitysupport.gwu.edu}{disabilitysupport.gwu.edu}
    to establish eligibility and to coordinate reasonable accommodations.
  \item \textbf{Student Health Center} \\
    202-994-5300\\
    The Student Health Center (SHC) offers medical,
    counseling/psychological, and psychiatric services to GW
    students. More information about the SHC is available at
    \href{https://healthcenter.gwu.edu}{healthcenter.gwu.edu}.
    Students experiencing a medical or mental health emergency on
    campus should contact GW Emergency Services at 202-994-6111, or
    off campus at 911.
\end{itemize}
\vspace{-.5em}

\subsection*{GW Campus Emergency Information}
GW Emergency Services: 202-994-6111
For situation-specific instructions, refer to
\href{https://safety.gwu.edu/emergency-response-handbook}{GW's
Emergency Procedures guide}.

\textbf{GW Alert}

GW Alert is an emergency notification system that sends alerts to the
GW community. GW requests students, faculty, and staff maintain
current contact information by logging on to
\href{https://alert.gwu.edu}{alert.gwu.edu}. Alerts are sent via
email, text, social media, and other means, including the Guardian
app. The Guardian app is a safety app that allows you to communicate
quickly with GW Emergency Services, 911, and other resources. Learn
more at \href{https://safety.gwu.edu}{safety.gwu.edu}.

\textbf{Protective Actions}

GW prescribes four protective actions that can be issued by
university officials depending on the type of emergency. All GW
community members are expected to follow directions according to the
specified protective action. The protective actions are Shelter,
Evacuate, Secure, and Lockdown (details below). Learn more at
\href{https://safety.gwu.edu/gw-standard-emergency-statuses}{safety.gwu.edu/gw-standard-emergency-statuses}.

\begin{itemize}
  \item Shelter
    \begin{itemize}
      \item Protection from a specific hazard.
      \item The hazard could be a tornado, earthquake, hazardous
        material spill, or other environmental emergency.
      \item Specific safety guidance will be shared on a case-by-case basis.
      \item \textbf{Action:}
        \begin{itemize}
          \item Follow safety guidance for the hazard.
        \end{itemize}
    \end{itemize}
  \item Evacuate
    \begin{itemize}
      \item Need to move people from one location to another.
      \item Students and staff should be prepared to follow specific
        instructions given by first responders and University officials.
      \item \textbf{Action:}
        \begin{itemize}
          \item Evacuate to a designated location.
          \item Leave belongings behind.
          \item Follow additional instructions from first responders.
        \end{itemize}
    \end{itemize}
  \item Secure
    \begin{itemize}
      \item Threat or hazard outside of buildings or around campus.
      \item Increased security, secured building perimeter, increased
        situational awareness, and restricted access to entry doors.
      \item \textbf{Action:}
        \begin{itemize}
          \item Go inside and stay inside.
          \item Activities inside may continue.
        \end{itemize}
    \end{itemize}
  \item Lockdown
    \begin{itemize}
      \item Threat or hazard with the potential to impact individuals
        inside buildings.
      \item Room-based protocol that requires locking interior doors,
        turning off lights, and staying out of sight of corridor window.
      \item \textbf{Action:}
        \begin{itemize}
          \item Locks, lights, out of sight.
          \item Consider Run, Hide, Fight.
        \end{itemize}
    \end{itemize}
\end{itemize}

\textbf{Classroom emergency lockdown buttons}

Some classrooms have been equipped with classroom emergency lockdown
buttons. If the button is pushed, GWorld Card access to the room will
be disabled, and GW Dispatch will be alerted. The door must be
manually closed if it is not closed when the button is pushed. Anyone
in the classroom will be able to exit, but no one will be able to get in.

\subsection*{Course Outline}

Readings with an embedded link can be accessed online. All other
readings are available on Blackboard.

\begin{tabularx}{\textwidth}{|p{.06\textwidth}p{.94\textwidth}|}
  \hline
  \textbf{Week} & \textbf{Course Material} \\

  %%% WEEK 1 %%%

  \hline
  \multirow{18}{*}{1} &

  January 12: Introduction \newline \newline
  \ul{Homework (all for Jan. 14)} \newline
  $\bullet$ Brown (2025),
  \href{https://reason.com/2025/08/18/is-conscientiousness-cratering-it-depends-on-how-you-twist-the-data/}{``Is
    Conscientiousness Cratering? It Depends on How You Twist the
  Data.''} (Reason) for class on Sept. 3 \newline
  $\bullet$ Dattani (2024),
  \href{https://ourworldindata.org/rise-us-maternal-mortality-rates-measurement}{``The
    rise in reported maternal mortality rates in the US is largely due
  to a change in measurement''} (Our World in Data) \newline
  $\bullet$ Steier (2025),
  \href{https://www.nytimes.com/interactive/2025/08/19/opinion/vaccines-autism-evidence.html?unlocked_article_code=1.fU8.g3du.FAXwl-X3uBVo\&smid=url-share}{``The
  Playbook Used to `Prove' Vaccines Cause Autism''} (New York Times
  Opinion) \newline \\
  \cdashline{2-2}
  & January 14: Researcher Choices \newline \newline
  \ul{Homework (all for Jan. 21)} \newline
  $\bullet$ Fry (2021), ``When Graphs Are a Matter of Life and
  Death'' (The New Yorker) (on Blackboard) \newline
  $\bullet$ Bugden (2019),
  \href{https://chezvoila.com/blog/warmingstripes/}{``Do you really
  understand the influential warming stripes?''} \newline
  $\bullet$ Watch the video lecture on Google Colab \newline \\

  %%% WEEK 2 %%%

  \hline
  & January 19: No Class (MLK, Jr. Day) \newline \\
  \cdashline{2-2}
  \multirow{5}{*}{2} & January 21: Data Visualization \newline \newline
  \ul{Homework} \newline
  $\bullet$ Watch the video lecture on data visualization for class
  on Jan. 26 \newline \\

  %%% WEEK 3 %%%

  \hline
  \multirow{11}{*}{3} &

  \textbf{January 26: Lab Day - data visualization} \newline \newline
  \ul{Homework} \newline
  $\bullet$ Complete lab assignment by Jan. 27 at 11:59pm \newline
  $\bullet$ Complete CITI Research Ethics training by Jan. 27 at
  11:59pm \newline \\
  \cdashline{2-2}
  & January 28: Data Ethics \newline \newline
  \ul{Homework} \newline
  $\bullet$ Watch the video lecture on data wrangling for class on
  Feb. 2 \newline \\

  %%% WEEK 4 %%%

  \hline
  \multirow{11}{*}{4} &

  \textbf{February 2: Lab Day - data wrangling} \newline \newline
  \ul{Homework} \newline
  $\bullet$ Complete lab assignment by Feb. 3 at 11:59pm \newline
  $\bullet$ Aviv (2024), ``Conviction'' (New Yorker) (on Blackboard)
  for class on Feb. 4 \newline \\
  \cdashline{2-2}
  & February 4: Correlation vs. Causation \newline \newline
  \ul{Homework} \newline
  $\bullet$ None \newline \\

  %%% WEEK 5 %%%

  \hline
  \multirow{13}{*}{5} &

  February 9: Sampling \newline \newline
  \ul{Homework} \newline
  $\bullet$ Excerpt of Morris (2022), \emph{Strength in Numbers} (on
  Blackboard) for class on Feb. 11 \newline \\
  \cdashline{2-2}
  & February 11: Political Polling I \newline \newline
  \ul{Homework (all for Feb. 18)} \newline
  $\bullet$ Excerpt of
  \href{https://aapor.org/wp-content/uploads/2025/10/AAPOR-Task-Force-on-2024-Pre-Election-Polling_Report.pdf}{AAPOR's
  Task Force on 2024 Pre-Election Polling Report} \newline
  $\bullet$ Morris (2025), ``The best pollsters of 2024 are doing a
  lot of things that just don't add up'' (Strength in Numbers) (on
  Blackboard) \newline \\

  %%% WEEK 6 %%%

  \hline
  &
  February 16: No Class (Presidents' Day) \newline \newline \newline \\
  \cdashline{2-2}
  \multirow{5}{*}{6} & February 18: Political Polling II \newline \newline
  \ul{Homework} \newline
  TBD \newline \\

  % %%% WEEK 7 %%%

  % \hline
  % \multirow{10}{*}{7} &

  % October 6: Correlation vs. Causation \newline \newline
  % \ul{Homework} \newline
  % $\bullet$ Excerpt of Reinhart (2015), \ul{Statistics Done Wrong}
  % (on Blackboard) for Oct. 8 \newline \\
  % \cdashline{2-2}
  % & October 8: Hypothesis testing \newline \newline
  % \ul{Homework} \newline
  % $\bullet$ Watch the video lecture on analyzing political polls
  % for class on Oct. 13 \newline \\

  % %%% WEEK 8 %%%

  % \hline
  % \multirow{14}{*}{8} &

  % \textbf{October 13: Lab Day - political polling} \newline \newline
  % \ul{Homework} \newline
  % $\bullet$ Complete lab assignment by Oct. 14 at 11:59pm \newline
  % $\bullet$ Excerpt of Tetlock and Gardner (2015),
  % \ul{Superforecasting: The Art and Science of Prediction} (on
  % Blackboard) for class on Oct. 15 \newline \\
  % \cdashline{2-2}
  % & October 15: Predictive Election Models \newline \newline
  % \ul{Homework} \newline
  % $\bullet$ Cools, Hannes, and Michael Koliska. 2024. ``News
  % Automation and Algorithmic Transparency in the Newsroom: The Case
  % of the Washington Post.'' \underline{Journalism Studies} 25(6):
  % 662–80. (on Blackboard) for class on Oct. 20 \newline \\

  % %%% WEEK 9 %%%

  % \hline
  % \multirow{11}{*}{9} &

  % October 20: Machine Learning \newline \newline
  % \ul{Homework} \newline
  % $\bullet$ None \newline \\
  % \cdashline{2-2}
  % & \textbf{October 22: Guest Speaker - G. Elliott Morris, author
  % of \textit{Strength in Numbers} and former Editorial Director of
  % \textit{FiveThirtyEight}} \newline \newline
  % \ul{Homework} \newline
  % $\bullet$ Watch the video lecture on hypothesis testing for class
  % on Oct. 27 \newline \\

  % %%% WEEK 10 %%%

  % \hline
  % \multirow{10}{*}{10} &

  % \textbf{October 27: Lab Day - hypothesis testing} \newline \newline
  % \ul{Homework} \newline
  % $\bullet$ Complete lab assignment by Oct. 28 at 11:59pm  \newline \\
  % \cdashline{2-2}
  % & October 29: Regression \newline \newline
  % \ul{Homework} \newline
  % $\bullet$ None  \newline \\

  % %%% WEEK 11 %%%

  % \hline
  % \multirow{10}{*}{11} &

  % \textbf{November 3: Lab Day - class polling project} \newline \newline
  % \ul{Homework} \newline
  % $\bullet$ Complete lab assignment by Nov. 4 at 11:59pm  \newline \\
  % \cdashline{2-2}
  % & November 5: Qualitative Data \newline \newline
  % \ul{Homework} \newline
  % $\bullet$ Watch the video lecture on text-as-data for class on
  % Nov. 10 \newline \\

  % %%% WEEK 12 %%%

  % \hline
  % \multirow{11}{*}{12} &

  % \textbf{November 10: Lab Day - text-as-data} \newline \newline
  % \ul{Homework} \newline
  % $\bullet$ Complete lab assignment by Nov. 11 at 11:59pm \newline \\
  % \cdashline{2-2}
  % & \textbf{November 12: Guest Speaker - John Kropf, fmr. Deputy
  % Chief Privacy Officer, U.S. Department of Homeland Security}
  % \newline \newline
  % \ul{Homework} \newline
  % $\bullet$ None \newline \\

  % %%% WEEK 13 %%%

  % \hline
  % \multirow{11}{*}{13} &

  % \textbf{November 17: Guest Speaker - Clayton Perry, Data
  % Scientist at Blue Rose Research} \newline \newline
  % \ul{Homework} \newline
  % $\bullet$ None \newline \\
  % \cdashline{2-2}
  % & \textbf{November 19: Aidan Hughes, Reporter at Inside Climate
  % News} \newline \newline
  % \ul{Homework} \newline
  % $\bullet$ Watch the video lecture on maps for class on Dec. 1 \newline \\

  % %%% WEEK 13 %%%

  % \hline
  % \multirow{10}{*}{14} &

  % \textbf{December 1: Lab Day - maps} \newline \newline
  % \ul{Homework} \newline
  % $\bullet$ Complete lab assignment by Dec. 2 at 11:59pm \newline \\
  % \cdashline{2-2}
  % & December 3: Potpourri \newline \newline
  % \ul{Homework} \newline
  % $\bullet$ None \newline \\

  % %%% WEEK 15 %%%

  % \hline
  % \multirow{1}{*}{15} &

  % December 8: Putting It All Together \\

  \hline

\end{tabularx}

The schedule for the remainder of the semester will be provided once
guest speaker dates are confirmed. Additional course topics that will
be covered include:
\begin{itemize}
  \item Predictive election models
  \item Machine learning
  \item Qualitative data
  \item Hypothesis testing and regression
  \item Data storytelling
\end{itemize}

\centering
Version: 1\\
Last Updated: January 11, 2026 \\
Subject to change.

\end{document}
