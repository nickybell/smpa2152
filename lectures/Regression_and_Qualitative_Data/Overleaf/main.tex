% The dvipsnames option is passed to the xcolor package, which beamer loads
\documentclass[xcolor={dvipsnames}]{beamer}

\usepackage{smpa2152-style}

\title[Regression \& Qual]{Regression and Qualitative Data}
\author[SMPA 2152]{Data Analysis for Journalism and Political Communication (Fall 2025)}
\date{Prof. Bell}

\begin{document}

%%%%%%%%%%%%%%%%%%%%%%%%%%%%%%%%%%%%%%%%%%%%%%%%%%%%%%%%%%%%%%%%%%
\frame{
\titlepage
}

%%%%%%%%%%%%%%%%%%%%%%%%%%%%%%%%%%%%%%%%%%%%%%%%%%%%%%%%%%%%%%%%%%
\frame{\frametitle{What is Regression?}

\begin{itemize}[<+->]
    \item So far, we've learned how to compare the mean of two groups using a \textbf{t-test}
    \item But often, a t-test is too restrictive for the analysis we want to conduct:
    \begin{itemize}
        \item What if we have two continuous variables? Income, age, and years of education are common variables that we may not want to force into two discrete categories.
        \item What about \textbf{confounders}? ~\\
        \begin{block}{Confounder}
        A variable that explains change in both the independent and dependent variable.
        \end{block}
        \item When we fail to account for confounders, we face \textbf{omitted variable bias} and our estimates are not accurate.
    \end{itemize}
\end{itemize}
}

%%%%%%%%%%%%%%%%%%%%%%%%%%%%%%%%%%%%%%%%%%%%%%%%%%%%%%%%%%%%%%%%%%
\frame{\frametitle{In-Class Example}
\Large \textbf{Linear regression is just the best fit line through a scatter plot of two or more variables.}
}

%%%%%%%%%%%%%%%%%%%%%%%%%%%%%%%%%%%%%%%%%%%%%%%%%%%%%%%%%%%%%%%%%%
\frame{\frametitle{Estimating the Regression (Best Fit) Line}
\only<1-3>{The linear regression equation is:
\begin{center}
$Y = \beta_0 + \beta_1X + \epsilon$\\
~\\
$Y$ = dependent variable\\
$\beta_0$ = intercept\\
$\beta_1$ = slope, also called a coefficient\\
$X$ = independent variable\\
$\epsilon$ = error
\end{center}}

\only<2-3>{This looks very similar to a linear equation you might have learned before:
\begin{center}
$y = ax + b$\\
\only<3>{$y = b + ax$}
\end{center}}
}

%%%%%%%%%%%%%%%%%%%%%%%%%%%%%%%%%%%%%%%%%%%%%%%%%%%%%%%%%%%%%%%%%%
\frame{\frametitle{In-Class Example}
\Large \textbf{The lm() function calculates the intercept and slope for the linear regression equation.}
}

%%%%%%%%%%%%%%%%%%%%%%%%%%%%%%%%%%%%%%%%%%%%%%%%%%%%%%%%%%%%%%%%%%
\frame{\frametitle{In-Class Example}

You want to estimate the effect of income on the allocation to welfare applicants. The linear regression equation is:
\begin{center}
$\hat{Allocation} = \hat{\beta}_0 + \hat{\beta}_1Income$
\end{center}

The \enspace  $\hat{}$ \enspace is the mathematical notation for ``estimate of the mean''\\
~\\
\only<2->{If ``Income'' in thousands has a value of 100, and the intercept is \$605, and the coefficient is -.5, what is our estimated mean of ``Allocation''?}

\only<3->{\begin{center}
$\hat{Allocation} = 605 + (-.5)*100$
\end{center}}

\only<4->{\begin{center}
$\hat{Allocation} = 555$
\end{center}}

\only<5>{So the coefficient ($\beta_1$) is the effect of a \textbf{one-unit} change in income (thousands) on the mean allocation.}
}

%%%%%%%%%%%%%%%%%%%%%%%%%%%%%%%%%%%%%%%%%%%%%%%%%%%%%%%%%%%%%%%%%%
\frame{\frametitle{What Is Error?}
\only<1>{Recall that the linear regression equation is:
\begin{center}
$Y = \beta_0 + \beta_1X + \epsilon$\\
~\\
$Y$ = dependent variable\\
$\beta_0$ = intercept\\
$\beta_1$ = slope, also called a coefficient\\
$X$ = independent variable\\
$\epsilon$ = error
\end{center}}
\only<3>{\centering
\includegraphics[width=.8\textwidth]{RegressionError.png}}
\only<2,4->{\begin{itemize}[<+->]
    \item Error ($\epsilon$ or $u$) is also called the residual (left over from $Y = \beta_0 + \beta_1X$, our best fit line)
    \item<3-> Our goal in regression is to fit the best line that  minimizes the error
    \item<5> However, we can never get the $\epsilon$ = 0, and often we don't even get close. We just do the best we can to make the ``best'' best fit line.
\end{itemize}}
}

%%%%%%%%%%%%%%%%%%%%%%%%%%%%%%%%%%%%%%%%%%%%%%%%%%%%%%%%%%%%%%%%%%
\frame{\frametitle{Multiple Linear Regression}
\only<1,3-4>{The multiple linear regression equation is:
\begin{center}
$Y = \beta_0 + \beta_1X_1 + \beta_2X_2 + \epsilon$\\
~\\
$Y$ = dependent variable\\
$\beta_0$ = intercept\\
$\beta_i$ = slope coefficient\\
$X_i$ = independent variable\\
$\epsilon$ = error
\end{center}
\only<3>{\textcolor{blue}{What is the interpretation of $\beta_1$?}\\
~\\
$\beta_1$ is the effect of a one-unit change in $X_1$ on the mean of $Y$, holding $X_2$ constant (the independent effect of $X_1$ on $Y$)}
\only<4>{\textcolor{blue}{What is the interpretation of $\beta_2$?}\\
~\\
$\beta_2$ is the effect of a one-unit change in $X_2$ on the mean of $Y$, holding $X_1$ constant (the independent effect of $X_2$ on $Y$)}}
\only<2>{
\centering
\includegraphics[height = .8\textheight]{3dregression.png}
}
}

%%%%%%%%%%%%%%%%%%%%%%%%%%%%%%%%%%%%%%%%%%%%%%%%%%%%%%%%%%%%%%%%%%
\frame{\frametitle{Non-Continuous Dependent Variables}
\begin{itemize}[<+->]
    \item One of the assumptions of linear regression is that the relationship between $X$ and $Y$ is linear (you can draw a best fit line)
    \item This assumption is usually fine when we are working with continuous dependent variables like income
    \item What about categorical dependent variables, like party ID? The linear regression model is not well suited for these.
    \item What about binary dependent variables, like support for a policy?\\
    ~\\
    We call this a linear probability model because it estimates the effect of a one-unit change in $X$ on the \textit{probability} (percent chance) of a value of ``1'' for the dependent variable
\end{itemize}
}

%%%%%%%%%%%%%%%%%%%%%%%%%%%%%%%%%%%%%%%%%%%%%%%%%%%%%%%%%%%%%%%%%%
\frame{\frametitle{In-Class Example}
}

%%%%%%%%%%%%%%%%%%%%%%%%%%%%%%%%%%%%%%%%%%%%%%%%%%%%%%%%%%%%%%%%%%
\frame{\frametitle{Qualitative Data Analysis}
\begin{itemize}[<+->]
    \item So far, we've focused on quantitative data — things we can count
    \item But what about data that isn't numbers? This is \textbf{qualitative data}
    \item There are at least twenty different types of qualitative data analysis, including several found in the social sciences:
    \begin{itemize}
        \item Case studies
        \item Ethnography
        \item Content analysis
        \item Mixed-methods
    \end{itemize}
    \item Our goal isn't to find a ``best fit line,'' but to find patterns, themes, and meaning from ``thick descriptions'' (Geertz, 1973)
    \item We will focus on one type of qualitative analysis: text analysis
\end{itemize}
}

%%%%%%%%%%%%%%%%%%%%%%%%%%%%%%%%%%%%%%%%%%%%%%%%%%%%%%%%%%%%%%%%%%
\frame{\frametitle{Text Analysis}
\only<1-6,8>{
    \begin{itemize}[<+->]
        \item Think of:
        \begin{itemize}
            \item<.-> Interview transcripts
            \item Open-ended survey responses
            \item News articles or social media posts
            \item Focus group notes
        \end{itemize}
        \item The most common way to analyze qualitative data is \textbf{thematic coding}
        \item A \textbf{code} is just a descriptive label applied to a piece of text
        \item We can then quantify our codes: how frequently each code appears, in which parts of the text, from which speakers, etc.
    \end{itemize}
}
\only<7>{
    \centering
    \includegraphics[width = \textwidth]{process_coding_huberman_etal.png}
}
}

%%%%%%%%%%%%%%%%%%%%%%%%%%%%%%%%%%%%%%%%%%%%%%%%%%%%%%%%%%%%%%%%%%
\frame{\frametitle{Where Do Codes Come From?}
\begin{itemize}[<+->]
    \item \textbf{Deductive Coding:} You start with a list of codes \textit{before} you read
    \begin{itemize}
        \item<.-> This is most often used when the goal of the study is to count the appearances of a discrete set of themes, e.g., "How often does Donald Trump talk about trade in his speeches?"
    \end{itemize}
    \item \textbf{Inductive Inductive:} You start \textit{without} a list fo codes. You read the data and let the themes emerge from the text
    \begin{itemize}
        \item<.-> Many qualitative research studies do not begin with well-defined hypotheses, and researchers generate new ideas through the process of coding
    \end{itemize}
    \item Most studies begin with a short list of deductive codes, then refine those codes or add new ones as you explore the data
\end{itemize}
}

%%%%%%%%%%%%%%%%%%%%%%%%%%%%%%%%%%%%%%%%%%%%%%%%%%%%%%%%%%%%%%%%%%
\frame{\frametitle{In-Class Exercise}

\textit{\textbf{From StoryCorps}}:\\
~\\
In February of 2012, Jamal Faison was a 20-year-old college sophomore home on school break in New York City when he, along with two others, were arrested for attempting to steal mobile devices from a subway rider. Transit police arrested Jamal and he spent the next eight months on Rikers Island — New York City’s massive main jail complex.\\
~\\
In September 2012, Jamal pleaded guilty to grand larceny and attempted robbery charges and a month later was released from custody. Jamal came to StoryCorps to remember the night he was released from Rikers, and to discuss his relationship with his uncle, Born.
}

%%%%%%%%%%%%%%%%%%%%%%%%%%%%%%%%%%%%%%%%%%%%%%%%%%%%%%%%%%%%%%%%%%
\frame{\frametitle{What If We Have Too Much Text?}
\begin{itemize}[<+->]
    \item Thematic coding by hand is great for 20 interviews, but what if we have 20,000 news articles? 2 million tweets?
    \item This is where computational text analysis comes in
    \item When working with computational text analysis, we call the full set of text and documents the \textbf{corpus}
    \item The techniques we will learn in this class are called ``bag of words'' techniques, because unlike an LLM, they are agnostic about the order of the words
    \begin{itemize}
        \item As if you are putting all the words into a bag, drawing them out at random, and seeing how frequently they occur
    \end{itemize}
\end{itemize}
}

%%%%%%%%%%%%%%%%%%%%%%%%%%%%%%%%%%%%%%%%%%%%%%%%%%%%%%%%%%%%%%%%%%
\frame{\frametitle{LDA Topic Modeling}
\begin{itemize}[<+->]
    \item \textbf{LDA} (Latent Dirichlet Allocation) is a popular method
    \item It is an unsupervised method: you don't give it a list of codes. It finds the themes (called ``topics'') made up of words that tend to appear together frequently
    \item The researcher must decide what the semantic meaning of these topics (correlated words) mean
\end{itemize}
}

%%%%%%%%%%%%%%%%%%%%%%%%%%%%%%%%%%%%%%%%%%%%%%%%%%%%%%%%%%%%%%%%%%
\frame{\frametitle{Researcher Choices in LDA}
\begin{enumerate}[<+->]
    \item How many topics?
    \begin{itemize}
        \item You must pre-define how many topics to create
        \item Too few: The topics might be too broad and meaningless
        \item Too many: The topics might be too specific, overlap, or be junk
        \item This is a critical choice with no single right answer
    \end{itemize}
    \item What do the topics mean?
    \begin{itemize}
        \item LDA just gives you a list of words: ``economy,'' ``jobs,'' ``growth...''
        \item Is that topic ``The Economy''? Or ``Labor Market Reports''? Or ``Political Talking Points''?
        \item The researcher must interpret and label the topic. This is a subjective, human step!
    \end{itemize}
\end{enumerate}
}

%%%%%%%%%%%%%%%%%%%%%%%%%%%%%%%%%%%%%%%%%%%%%%%%%%%%%%%%%%%%%%%%%%
\frame{\frametitle{In-Class Exercise}
\begin{itemize}
    \item Data at: \href{https://www.kaggle.com/datasets/christianlillelund/donald-trumps-rallies}{Kaggle.com}
    \item Tool: \href{https://voyant-tools.org/}{Voyant}
\end{itemize}

}

\end{document}
