% The dvipsnames option is passed to the xcolor package, which beamer loads
\documentclass[xcolor={dvipsnames}]{beamer}

\usepackage{smpa2152-style}

\title[Researcher Choices]{Researcher Choices}
\author[SMPA 2152]{Data Analysis for Journalism and Political
Communication (Spring 2026)}
\date{Prof. Bell}

\begin{document}

%%%%%%%%%%%%%%%%%%%%%%%%%%%%%%%%%%%%%%%%%%%%%%%%%%%%%%%%%%%%%%%%%%
\frame{
  \titlepage
}

%%%%%%%%%%%%%%%%%%%%%%%%%%%%%%%%%%%%%%%%%%%%%%%%%%%%%%%%%%%%%%%%%%
\frame{\frametitle{Bellringer}
  \inserttimer{5}
}

%%%%%%%%%%%%%%%%%%%%%%%%%%%%%%%%%%%%%%%%%%%%%%%%%%%%%%%%%%%%%%%%%%
\frame{

  \begin{center}
    \includegraphics[width=.9\textwidth]{soccer_same_data_different_results.png}
  \end{center}
}

%%%%%%%%%%%%%%%%%%%%%%%%%%%%%%%%%%%%%%%%%%%%%%%%%%%%%%%%%%%%%%%%%%
\frame{\frametitle{Researcher Choices}

  \begin{enumerate}[<+->]
    \item What is my hypothesis?
    \item What do I measure?
    \item How do I collect my data?
    \item How much data do I collect?
    \item What statistical analyses do I use?
    \item How do I handle outliers, missing data, and other peculiarities?
  \end{enumerate}
}

%%%%%%%%%%%%%%%%%%%%%%%%%%%%%%%%%%%%%%%%%%%%%%%%%%%%%%%%%%%%%%%%%%
\frame{\frametitle{What is my hypothesis?}

  Choosing a hypothesis is all about avoiding error:\\~\\

  \begin{itemize}[<+->]
    \item \textbf{Type I error}: False positives
      \begin{itemize}
        \item<.-> Sending the innocent to jail - this is bad!\\~\\
      \end{itemize}
    \item \textbf{Type II error}: False negatives
      \begin{itemize}
        \item<.-> Letting the guilty go free - we can accept this
      \end{itemize}
  \end{itemize}
  ~\\
  \only<3->{Our goal is to reduce Type I error. Assume that the data
  is innocent (that the hypothesis is false) until it is proven guilty.}
}

%%%%%%%%%%%%%%%%%%%%%%%%%%%%%%%%%%%%%%%%%%%%%%%%%%%%%%%%%%%%%%%%%%
\frame{\frametitle{Measuring Type I Error}

  \only<1-3>{
    \begin{block}{P-value}
      The chance that we would get a particular result from our test
      if the true answer is false
    \end{block}

    \begin{itemize}[<+(1)->]
      \item The p-value is our chance of committing a Type I error -
        sending the innocent to jail
      \item Common p-value cut-offs in scientific research: .01, .05,
        and .1 indicate \textbf{statistical significance}
    \end{itemize}
  }
  \only<4->{
    \textbf{Hypothesis:} A student cheated on an exam.
    ~\\
    ~\\
    \begin{itemize}
      \item<5-> The student performed much better on this exam than
        on previous exams
      \item<6-> The student finished their exam more quickly than other students
      \item<7-> The student's roommate saw them up all night studying
        before the exam
      \item<8-> The student missed the same questions as other students
    \end{itemize}
    ~\\
    \textbf{P-value (chance that we conclude that student cheated,
    but they did not):}
    \only<4>{.50}\only<5>{.25}\only<6>{.15}\only<7>{.40}\only<8>{.75}
  }
}
%%%%%%%%%%%%%%%%%%%%%%%%%%%%%%%%%%%%%%%%%%%%%%%%%%%%%%%%%%%%%%%%%%
\frame{\frametitle{The Scientific Method}

  \begin{itemize}[<+->]
    \item P-values are a product of the data we use and our choices
      about what we include and exclude from the analysis.
    \item We follow the scientific method: Theory $\Rightarrow$
      Hypothesis $\Rightarrow$ Test $\Rightarrow$ Analyze $\Rightarrow$ Report
    \item But in practice, no analysis plan survives contact with the data
  \end{itemize}
}

%%%%%%%%%%%%%%%%%%%%%%%%%%%%%%%%%%%%%%%%%%%%%%%%%%%%%%%%%%%%%%%%%%
\frame{\frametitle{Are Democrats or Republicans Good for the Economy?}

  Use (a recreation of) FiveThirtyEight's online tool to test what
  you think is the best approach to answering the question. There are
  no right or wrong answers - just select the model you think is
  best, and report your results in the form:

  \begin{center}
    \href{https://bit.ly/smpa2152}{https://bit.ly/smpa2152}\\
    ~\\
    \includegraphics[height=.3\textheight]{bit.ly_smpa2152.png}
  \end{center}

  (The link to the tool is on the form.)
}

%%%%%%%%%%%%%%%%%%%%%%%%%%%%%%%%%%%%%%%%%%%%%%%%%%%%%%%%%%%%%%%%%%
\frame{\frametitle{Are Democrats or Republicans Good for the Economy?}
  \inserttimer{5}
}

%%%%%%%%%%%%%%%%%%%%%%%%%%%%%%%%%%%%%%%%%%%%%%%%%%%%%%%%%%%%%%%%%%
\frame{\frametitle{What do I measure?}

  \begin{block}{Operationalization}
    The process of defining a measurable version of a concept.
  \end{block}
  ~\\
  \begin{center}
    \includegraphics[width=.48\textwidth]{maternal_mortality.png}
    \hfill
    \includegraphics[width=.48\textwidth]{nyt_autism_traits.png}
  \end{center}
}

%%%%%%%%%%%%%%%%%%%%%%%%%%%%%%%%%%%%%%%%%%%%%%%%%%%%%%%%%%%%%%%%%%
\frame{\frametitle{Principles of Good Operationalization}
  \only<1-5>{
    \begin{enumerate}[<+(0)->]
      \item Unambiguous
      \item Parsimonious
      \item Accurate
      \item Reliable
      \item Feasible
    \end{enumerate}
    \only<1>{
      \begin{center}
        \href{https://www.youtube.com/watch?v=oLAtkkkUAPc}{\includegraphics[height=.6\textheight]{EnglishmanWentUpHillPoster.png}}
      \end{center}
    }
    \only<3>{
      \begin{center}
        \includegraphics[height=.6\textheight]{census_2010.jpg}
        \hspace{2em}
        \includegraphics[height=.6\textheight]{census_2030.png}
      \end{center}
    }
    \only<4>{
      \begin{center}
        \includegraphics[height=.6\textheight]{IAT.png}
      \end{center}
    }
  }
  \only<6>{
    \begin{block}{Index (or Scale) Variable}
      An index (or scale) variable is a type of proxy measure in
      which the researcher combines multiple sources of data into a
      single representation of the concept under investigation using
      a pre-defined mathematical operation.
    \end{block}
    \begin{center}
      \includegraphics[width=.6\textwidth]{gender_inequality_index.png}
    \end{center}
  }
}

%%%%%%%%%%%%%%%%%%%%%%%%%%%%%%%%%%%%%%%%%%%%%%%%%%%%%%%%%%%%%%%%%%
\frame{\frametitle{Exercise: Operationalization}

  \begin{enumerate}
    \item You want to measure
      \href{https://worldhappiness.report/}{how happy people are}
    \item You want to measure
      \href{https://www.dmv.virginia.gov/sites/default/files/forms/csma19.pdf}{people's
      driving ability}
    \item You want to measure the
      \href{https://voteview.com/congress/house}{political ideology
      of a member of Congress}
  \end{enumerate}
}

%%%%%%%%%%%%%%%%%%%%%%%%%%%%%%%%%%%%%%%%%%%%%%%%%%%%%%%%%%%%%%%%%%
\frame{\frametitle{How do I collect my data?}

  \begin{block}{Data Generating Process}
    The rules and procedures that produce the data one is interested in
  \end{block}

  \begin{itemize}[<+(1)->]
    \item No amount of statistical wizardry can compensate for bad data
    \item The gold standard of data generating processes is the
      \textbf{random sample}
  \end{itemize}

  \centering
  \vspace{1em}
  \only<2>{
    \includegraphics[width=.6\textwidth]{dgp_xkcd.png}
  }
}

%%%%%%%%%%%%%%%%%%%%%%%%%%%%%%%%%%%%%%%%%%%%%%%%%%%%%%%%%%%%%%%%%%
\frame{\frametitle{Sampling}
  \begin{itemize}[<+->]
    \item The group we are interested in studying is known as the
      \textbf{population}
    \item Often, we are not able to count every unit in the
      population, so we take a \textbf{sample}
    \item The key to any data analysis project is a quality sample,
      which is determined by two elements:
      \begin{enumerate}
        \item A \textbf{random sample} of the population
        \item The \textbf{sample size} is sufficiently large
      \end{enumerate}
  \end{itemize}
}

%%%%%%%%%%%%%%%%%%%%%%%%%%%%%%%%%%%%%%%%%%%%%%%%%%%%%%%%%%%%%%%%%%
\frame{\frametitle{Random Sample}

  \only<1-4>{
    \begin{block}{Definition}
      The probability of any given unit being drawn from the
      population is uniform (the same)
    \end{block}
    ~\\
    \begin{itemize}[<+(1)->]
      \item A failure of each unit to have a uniform probability of
        being drawn from the population is known as \textbf{selection bias}
      \item Units are ``selecting'' into our data because they are
        more observable than other units
      \item Selection bias reduces our \textbf{generalizability} to
        the population because the data is not representative of the population
    \end{itemize}
  }

  \only<5-7>{
    Can I randomly sample 10 students from this class to generalize to:
    \begin{itemize}[<+(4)->]
      \item the population of GW students?
      \item the population of SMPA students?
      \item the population of Data Analysis students?
    \end{itemize}
  }
}

%%%%%%%%%%%%%%%%%%%%%%%%%%%%%%%%%%%%%%%%%%%%%%%%%%%%%%%%%%%%%%%%%%
\frame{\frametitle{Exercise: Selection Bias}
  \inserttimer{5}
}

\end{document}
