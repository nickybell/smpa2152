% The dvipsnames option is passed to the xcolor package, which beamer loads
\documentclass[xcolor={dvipsnames}]{beamer}

\usepackage{smpa2152-style}

\title[Correlation vs. Causation]{Correlation vs. Causation}
\author[SMPA 2152]{Data Analysis for Journalism and Political
Communication (Spring 2026)}
\date{Prof. Bell}

\begin{document}

%%%%%%%%%%%%%%%%%%%%%%%%%%%%%%%%%%%%%%%%%%%%%%%%%%%%%%%%%%%%%%%%%%
\titlegraphic{\includegraphics[width=.6\textwidth]{xkcd_correlation.png}}
\frame{
  \titlepage
}

%%%%%%%%%%%%%%%%%%%%%%%%%%%%%%%%%%%%%%%%%%%%%%%%%%%%%%%%%%%%%%%%%%
\frame{\frametitle{"Conviction" by Rachel Aviv (New Yorker)}
  \centering
  \includegraphics[width=\textwidth]{letby.png}
}

%%%%%%%%%%%%%%%%%%%%%%%%%%%%%%%%%%%%%%%%%%%%%%%%%%%%%%%%%%%%%%%%%%
\frame{
  \only<1-2,4>{
    \begin{block}{Correlation}
      Correlation exists when the absolute rate of change in the
      values of two variables are similar.
    \end{block}
    ~\\
    \begin{itemize}
      \item<2-> \textbf{Positive correlation}: As the value of one
        variable increases (decreases), the value of the other
        variable increases (decreases) at the same rate
      \item<4-> \textbf{Negative correlation}: As the value of one
        variable increases (decreases), the value of the other
        variable decreases (increases) at the same rate
    \end{itemize}
  }
  \only<3>{
    \centering
    \includegraphics[width=.9\textwidth]{pos_correlation.png}
  }
  \only<5>{
    \centering
    \includegraphics[width=.9\textwidth]{neg_correlation.png}
  }
}

%%%%%%%%%%%%%%%%%%%%%%%%%%%%%%%%%%%%%%%%%%%%%%%%%%%%%%%%%%%%%%%%%%
\frame{
  \only<1-4>{
    \begin{block}{Causation}
      Causation exists when a change in the value of one variable
      would not be observed without a preceding change in the value
      of another variable.
    \end{block}
    ~\\
    \begin{itemize}[<+(1)->]
      \item Correlation is descriptive, while causation is predictive
      \item We say that the \textbf{independent} or
        \textbf{explanatory} variable causes the \textbf{dependent}
        or \textbf{outcome} variable
      \item Causation depends on knowing the \textbf{counterfactual}:
        if we did not observe a change in the value of the
        explanatory variable, we would not observe a change in the
        value of the outcome variable
    \end{itemize}
  }

  \only<5>{
    \centering
    \includegraphics[width=.8\textwidth]{nyt_public_opinion.png}
  }
  \only<6>{
    \centering
    \includegraphics[width=.8\textwidth]{2024polls.png}
  }
  \only<7>{
    ``15 Lessons Scientists Learned About Us When the World Stood
    Still'' (\textit{New York Times})
    \begin{enumerate}
      \item Home-field advantage is real (when fans are there).
      \item Women who rest more and were exposed to fewer stressors,
        pollutants and viruses have fewer premature births.
      \item Dolphins whistle longer, birds change their songs, and
        sea turtles lay more eggs when humans aren't around.
    \end{enumerate}
  }
}

%%%%%%%%%%%%%%%%%%%%%%%%%%%%%%%%%%%%%%%%%%%%%%%%%%%%%%%%%%%%%%%%%%
\frame{
  \begin{block}{The Fundamental Problem of Causal Inference}
    For any given case, we observe the outcome variable with
    \textit{either} a change in the independent variable or no change
    in the independent variable, but not both.
  \end{block}
  ~\\
  \begin{itemize}[<+(1)->]
    \item In other words, we do not observe the counterfactual
    \item But we try out best to observe the counterfactual using
      \textbf{experiments}
    \item Experiments establish a counterfactual by comparing cases
      that differ only in the explanatory variable that we are interested in
  \end{itemize}
}

%%%%%%%%%%%%%%%%%%%%%%%%%%%%%%%%%%%%%%%%%%%%%%%%%%%%%%%%%%%%%%%%%%
\frame{
  \only<1>{
    \centering
    \includegraphics[height=.9\textheight]{plant_experiment.png}
  }

  \only<2>{\frametitle{John Snow's Cholera Experiment}
    \centering
    \includegraphics[width=.8\textwidth]{snow_map.jpg}
  }

  \only<3>{\frametitle{John Snow's Cholera Experiment}
    \centering
    \includegraphics[width=.8\textwidth]{snow_results.png}
  }
}

%%%%%%%%%%%%%%%%%%%%%%%%%%%%%%%%%%%%%%%%%%%%%%%%%%%%%%%%%%%%%%%%%%
\frame{\frametitle{Experiments}

  \only<1-3>{
    \begin{itemize}[<+->]
      \item A critical element of experiments is that the cases are
        assigned to the \textbf{treatment} group (e.g., getting a
        drug) and \textbf{control} group (e.g., getting a placebo)
        completely at random
      \item Recall the definition of a \textbf{random sample}: the
        probability of any given unit being drawn from the population
        is uniform (the same)
      \item The intuition for randomness is that there is no
        \textbf{selection bias}: patients aren't getting the drug
        because they are younger or healthier, for example.
    \end{itemize}
  }
  \only<4-5, 7-8>{
    \begin{itemize}[<+(3)->]
      \item But wait: were households in London assigned to either
        the Southwark \& Vauxhall Company or the Lambeth Company
        completely at random?
      \item No. This is called a natural experiment, and it relies on
        whether we believe being in either group is as-good-as-random.
      \item<7-> Finding natural experiments in the real world is
        really difficult, so we often design our experiments in
        controlled settings like laboratories or surveys
      \item<8-> We only have a true experiment where the
        \underline{researcher} randomly assigns cases to treatment or control.
    \end{itemize}
  }

  \only<6>{
    \centering
    Card and Krueger (1994)\\
    \includegraphics[width=.9\textwidth]{card_krueger.png}
  }

  \only<9>{
    \centering
    Kam and Zechmeister (2013)\\
    \includegraphics[width=.6\textwidth]{kam-zechmeister.jpg}
  }

}

%%%%%%%%%%%%%%%%%%%%%%%%%%%%%%%%%%%%%%%%%%%%%%%%%%%%%%%%%%%%%%%%%%
\frame{\frametitle{Assessing Causality Without Experiments}

  \begin{itemize}[<+->]
    \item What about when we don't have an experiment, but instead
      are collecting \textbf{observational data} from the world?
    \item Then we face the fundamental problem of causal inference.
      Correlation does not imply causation.
    \item But all hope is not lost - we just have to be much more
      careful about threats to causal inference: \textbf{confounders}
      and \textbf{reverse causation}.
  \end{itemize}
}

%%%%%%%%%%%%%%%%%%%%%%%%%%%%%%%%%%%%%%%%%%%%%%%%%%%%%%%%%%%%%%%%%%
\frame{\frametitle{Confounders}
  \only<1-3>{
    \begin{block}{Confounder}
      A variable that explains change in both the independent and
      dependent variable.
    \end{block}
    \vspace{1em}
  }

  \only<2>{
    \centering
    \includegraphics[width=.9\textwidth]{causal_diagram.png}
  }
  \only<3>{
    \centering
    \includegraphics[width=.8\textwidth]{confounding.png}
  }
  \only<4>{
    \centering
    \includegraphics[width=.9\textwidth]{news_and_sentiment.jpeg}
  }
}

%%%%%%%%%%%%%%%%%%%%%%%%%%%%%%%%%%%%%%%%%%%%%%%%%%%%%%%%%%%%%%%%%%
\frame{\frametitle{Reverse Causation}
  \only<1-3>{
    \begin{block}{Reverse Causation}
      Occurs when the dependent variable affects the independent variable.
    \end{block}
    \vspace{1em}
  }

  \only<2>{
    \centering
    \includegraphics[width=.9\textwidth]{causal_diagram.png}
  }
  \only<3>{
    \centering
    \includegraphics[width=.9\textwidth]{reverse_causation.png}
  }
  \only<4>{
    \centering
    \includegraphics[width=.8\textwidth]{consumer_sentiment.png}
  }
}

%%%%%%%%%%%%%%%%%%%%%%%%%%%%%%%%%%%%%%%%%%%%%%%%%%%%%%%%%%%%%%%%%%
\frame{\frametitle{Exercise: Correlation vs. Causation}
  \inserttimer{6}
}

\end{document}
